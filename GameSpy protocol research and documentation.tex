\documentclass[oneside,titlepage,a4paper]{Definition/retrospy} %book,article,report,letter


\begin{document}
 
% Header

\title{\Huge\textbf{GameSpy protocol research and documentation}} 
\author{Arves100, Xiaojiuwo}

\maketitle

\tableofcontents

%\mainmatter %%表示文章的正文部分,在生成目录后将从第一页开始

%\part{Protocol and server research} % We should divide the RetroSpy structure, game research from each other.
	
	\chapter{General Construction}
		\par Inside the GameSpy SDK there are 16 compotents, which constructed the GameSpy services.
		\subsection{SDK Module}
		\begin{itemize}
			\item Brigades
			\item Chat
			\item Presence \& Messaging
			\item CDKey
			\item Stats \& Tracking
			\item Persistent Storage
			\item Transport
			\item NAT Negotation
			\item Peer to Peer communication
			\item Patching \& Tracking
			\item Server Browser
			\item Query \& Reporting
			\item SAKE Persistent Storage
			\item ATLAS Competition
			\item Voice Chat
			\item Web Authentication
		\end{itemize}
		\subsection{Servers}
		\par Those compoents allows accessing multiple GameSpy servers that they provide.
		\begin{itemize}
			\item Presence Connection Manager
			\item Presence Search Player
			\item Master (Backend)
			\item Chat
			\item SAKE Storage
			\item NAT Negotation Mathup
			\item Server Browser
			\item CDKey
			\item Stats \& Tracking
			\item Web Services
		\end{itemize}
	\chapter{Presence Servers}
		            Presence \& Messaging system allows a game to add account authentication or registration, which includes a profile where personal information could be stored (such as email, first name) and a friend list (called buddies).
            The presence server also allows to set a location string, which could be used to indicate where the player is currently playing at (for example: GameSpy Arcade uses home://).

		\subsection{Account representation}
		GameSpy used to store accounts that are associated for a game in a specific way. In the presence server: User, Profile, Unique nickname and Namespace ID are dinstincted.
		\subsubsection{User}
	            A Presence account rapresents an User (the term used by the SDK), an user contains the EMail and Password for such user.
	            The Email address MUST BE unique, as it's used to identify an user in every context (for example: multiple games).
		\subsubsection{Profile}
	            A profile contains the information of an user regarding a game (or the friend list for such game). It could also store information such as birth date and surname.
	            An user could have multiple profiles, and a profile must be associated with only one user.
		\subsubsection{Namespace \& Unique Nicknames}
	            A namespace identify a space where a game is defined (for example: namespace 0 identify GameSpy Arcade).
	            In this context, an unique nickname could be associated to the user, such nickname <strong>MUST BE</strong> unique in that namespace.
	            An profile could have multiple unique nicknames for each namespace, but it could only have one unique nickname in such namespace.

		\subsection{Login methods}
		There are three methods that a client can login to the GPCM.
		\begin{itemize}
			\item Profile
			\item Unique nickname
			\item PreAuthenticated
			%\item CD-Key (Verify if it's possible to login by using a CD-Key)
		\end{itemize}

		\subsubsection{Profile}
			The profile login is the standard way to login to GPCM, it requires that the client sends the email, the profile nickname and the namespace identification number, which will be used to identify both the profile and the user.
			The server returns the profile identification number, the user identification number, and the unique nickname associated to that profile in the specific namespace context.
		\subsubsection{Unique nickname}
			The unique nickname login is used for loggin in with a special nickname in a special game context. Both the unique nickname and the namespace are used to identify the profile and the user.
			The server returns the profile identification number, the user identification number and the profile nickname associated to that unique nickname in the specific namespace context.
		\subsubsection{PreAuthenticated}
			The Pre-Authenticated method (also known as Authentication token method) is a special method of logging into GameSpy services that works in conjunction of the Web Services.
			Using this method, you don't login with a password, nickname or email. You login by using an authentication token and a partner challenge.
			This two could be retrived from the Web Service Server.
			The user that is logging in with this method is usually not an ordinary account, it's a "shadow" account, created just for the following backend login (and possibly game).
			This method is used by special third-party backends that saves the login information in a different way.
			An example of this is shown with PlayStation 3 login (NP), the Web Service would generate an auth token and partner challenge from the SEN/PSN User information.
			The client needs to send the authentication token and the partner challenge disguized as an MD5 password.
			The server returns the profile identification number, the user identification number, the profile nickname and the unique nickname associated to the specific namespace context.
			Keep in mind that this are permanent account associated with the backend, in case of PlayStation 3 service, you can also see your SEN/PSN friends in your GameSpy buddies list if they had created a shadow account for use in GameSpy services.

		%\subsubsection{CD-Key}

		\subsection{Login system}
			Passwords aren't sent in login requests, rather, both client and server generate a MD5 hash known as "proof", which has in the intent to verify if an user could authenticate to the server or not.
			The proof is not only used to verify the user password, but also to make sure that you're not connecting to a fake server, or that the password has not been tampred.
			It's the job of both client and server to verify the proof once it has been sent by the client or the server. (In the server response, it's known as response)
			If the proof is not verified, the server sends an error message and close the connection, the client directly cuts the connection with the server.

			\subsubsection{The login proof}
			The login proof is a 32 hexadecimal string which is used for different purposed,the method of generating it it's different, but yet similar, for both client and server.
			
			\paragraph{Variable user information:} Inside the proof, a variable user information (referenced as "user" in the SDK) will be generated. This user information changes for each of the three login method used.
				If you're logging in with the Profile method, the user information will contain "profile nickname"@"email".
				If you're logging in with the Unique nickname method, the user information will contain just the unique nickname.
				If you're logging in with the Pre-Authenticated method, the user information will contain the authentication token.
				If the Partner ID specified is different than 0 (GameSpy), it will append to the beginning of the user string the following data: "partner id number@". This is not applied to the PreAuthenticated method.
		
			\paragraph{Client generation:} First, an MD5 hash of the password is calculated (in case of the Pre-Authenticated method, it's the Partner challenge), then 48 white spaces are appended to the md5 hash password.
				Then, the variable user information, client challenge and server challenge are appended to the previous string. This long generated string is hashed with MD5 and expressed in hexadecimal.

			\paragraph{Server generation:} The server generation is almost the same as the client one, the only difference is that the server challenge is appended BEFORE the client challenge, and not after.
				
			\subsubsection{Login request and response}
			The login starts by the server, after a client has been accepted, the server sends a "lc" (login challenge) command with the value 1, containing it's id and the server challenge.
			This challenge will be used for generating the proof.
			
			The client answers by sending a "login" command containing the data required to login with it's login method and the client challenge and the client generated proof, that the server will verify.
			The server then, calculated it's proof and verify the client one, in case there is a mismatch, the server disconnects the client after sending an error (See /ref{GPCMErrors} for a list of the known errors).
			If the verification of the client proof has succeeded, the server sends a "lc" command with the value 2, containing the profile's unique nickname, nickname, the server proof and the session key.
			The clients now, needs to verify if the server proof matches the generated one, in case there is a mismatch, the client disconnects from the server.
		
		\newpage
		\subsection{Protocol information}
			The data in both the presence servers (Connection Manager and Search Player) are exchanged with ASCII strings that contains request \& response of the client and the server. This ASCII data is known as "Command".

			\subsubsection{Command Pattern}
			This type of string is represent map containing both a key and value.
			A command starts with the character "$ \backslash $" and it ends with the following data "$ \backslash final \backslash $".
			As an example, we provide two example strings, one from the client \ref{Client example command} and one from the server \ref{Server example command}

			\begin{tcolorbox}
				$ \backslash login \backslash \backslash challenge \backslash jfeuihj48t4h3w8wj8f34j8 \backslash user \newline
				\backslash spyguy@spyguy@gamespy.com \backslash response \newline \backslash fje4ifh482teifh348h7u \backslash port \newline
				\backslash -9805 \backslash productid \backslash 10469 \backslash gamename \newline \backslash conflictsopc \backslash namespaceid \backslash 1 \backslash id \backslash 1 \backslash final \backslash $
				\label{Client example command}
			\end{tcolorbox}

			\begin{tcolorbox}
				$ \backslash lc \backslash 1 \backslash challenge \backslash JFU78DCH78 \backslash id \backslash 1 \backslash final \backslash  $
				\label{Server example command}
			\end{tcolorbox}

			Each $ \backslash $ divide a key to it's value, for example: the first key is "lc" and the first value is "1".
			The first key is always the command name, for example "lc" is the command name of that server request, while "login" is the command name of the client request.
			The value after the command name could be used to decribe what action is performing at the moment, in the case of $ \backslash lc \backslash 1 \backslash $ it is perfoming the action 1 of the command "lc".
			Client usually doesn't send a value after the command name, leaving two $ \backslash $ after the command name.

			\subsubsection{Errors}
			The error command is the same for both the two presence server. The table \ref{GPErrors} describes the keys that the error string could have.
			
			\begin{table}[H]
				\centering
				\label{GPError}
				\caption{GP Error}
				\begin{tabular}{c|c}
					\hline
					\textbf{Name} & \textbf{Description} \\
					\hline
					error & This is the command name, it does not require any extra parameter \\
					\hline
					err & The ID of the error \\
					\hline
					errmsg & A string that describes the error \\
					\hline
				\end{tabular}
			\end{table}

		\newpage
		\subsection{Connection Manager Server}
		Connection Manager Server (also known as GPCM), is a server that manage the user authentication and profile information. The IP, port and protocol of the server are shown in the table \ref{GPCMInfo}.
		
		\begin{table}[H]
			\centering
			\begin{tabular}{c|c|c}
				\hline 
				\textbf{IP} & \textbf{Port} & \textbf{Protocol} \\ 
				\hline 
				gpcm.gamespy.com & 29900 & TCP only \\ 
			 	\hline
			\end{tabular} 
		\caption{IP \, Port and protcol of GPCM}
		\label{GPCMInfo}
	\end{table}

\begin{comment}

	\subsection{Request and responses}
	The Table \ref{GPCMRequests} lists the request (that we were able to found) that the client sends to GPCM.
\begin{table}[H]
	\centering
	\begin{tabular}[|c|c|]
		\hline 
		\textbf{Commands}&\textbf{Description}  \\ 
		\hline 
		$\backslash inviteto \backslash\backslash$& Invite friends\\ 		
		\hline 
		$\backslash login \backslash\backslash$&Login to GPCM \\
		\hline
 		$\backslash getprofile \backslash\backslash$&	Get profile of some player\\
 		\hline
		$\backslash addbuddy \backslash\backslash$& Add player to my friend list \\
		\hline
		$\backslash delbuddy \backslash\backslash$ & Delete player from my friend list \\
		\hline
		$\backslash updateui \backslash\backslash$& ?\\
		\hline
		$\backslash updatepro \backslash\backslash$& Update player's profile such as first name, last name, gender etc. \\
		\hline
		$\backslash logout \backslash\backslash$& Logout manually by user\\
		\hline
		$\backslash status \backslash\backslash$& Update status of a user\\
		\hline
		$\backslash ka \backslash\backslash$& Keep client alive(do not disconnect) \\
 		
		\hline 
	\end{tabular} 
	\caption{Request for GameSpy Presence Connection Manager}
	\label{GPCMRequests}
\end{table}

Error response string for (GPCM, GPSP):
\begin{equation}
\begin{split}
\backslash error \backslash\backslash err \backslash \langle error code \rangle \backslash fatal\backslash\backslash errmsg \backslash \langle error message \rangle \backslash id\backslash 1 \backslash final \backslash
\end{split}	
\end{equation}
\subsubsection{Login Phase}
\myparagraph{Client Login Request}
There are three ways of login:
\begin{itemize}
	\item AuthToken: Logging using an alphanumeric string that rapresents an user
	\item 	UniqueNick: Logging using a nickname that is unique from all the players
	\item User: Logging with the nickname and the password
\end{itemize}

The full login request string:
\begin{equation}\label{Chanllenge string}
\begin{split}
	&\backslash login \backslash challenge \backslash \langle challenge \rangle \backslash authtoken \backslash \langle authtoken \rangle \\& \backslash uniquenick \backslash \langle uniquenick \rangle \backslash user \backslash \langle user \rangle 
	\backslash userid \backslash \langle userid \rangle \\& \backslash profileid \backslash \langle profileid \rangle \backslash partnerid \backslash \langle partnerid \rangle \backslash response \backslash \langle response \rangle \\&
	 \backslash firewall \backslash 1 \backslash port \backslash \langle port \rangle \backslash productid \backslash  \langle productid \rangle \\& \backslash gamename \backslash \langle gamename \rangle \backslash namespaceid \backslash \langle namespaceid \rangle \\& \backslash  sdkrevision \backslash \langle sdkrevision \rangle \backslash quiet \backslash \langle quiet \rangle \backslash id \backslash 1 \backslash final \backslash
\end{split}
\end{equation}
The value $ \langle challenge \rangle $ for $ \backslash challenge \backslash $ in \ref{Chanllenge string} is a 10 byte alphanumeric string.

The following Table \ref{Login parameter string} is a description of string used in login request, GameSpy can use these string to find value in database.
\begin{table}[H]
	\centering
%\begin{tabular}{A}

	\begin{tabular}{A}
		\hline
		\textbf{Keys} & \textbf{Description} & \textbf{Type}
			                                                                          \\ \hline
		 login& The login command which use to identify the login request of client&\\ \hline
		 challenge  & The user challenge used to verify the authenticity of the client     &                                                                                                                                     \\ \hline
		 authtoken  & The token used to login (represent of an user)        &                                                                                                                                                    \\ \hline
		uniquenick  & The unique nickname used to login       &                                                                                                                                                                  \\ \hline
		   user     & The users account (format is NICKNAME@EMAIL)           &                                                                                                                                                   \\ \hline
		  userid    & Send the userid (for example when you disconnect you will keep this)              &                                                                                                                        \\ \hline
		 profileid  & Send the profileid (for example when you disconnect you will keep this)           &                                                                                                                        \\ \hline
		 partnerid  & This ID is used to identify a backend service logged with gamespy.(Nintendo WIFI Connection will identify his partner as 11, which means that for gamespy, you are logging from a third party connection) &\\ \hline
		 response   & The client challenge used to verify the authenticity of the client     &                                                                                                                                   \\ \hline
		 firewall   & If this option is set to 1, then you are connecting under a firewall/limited connection &\\
		 \hline
		 port& The peer port (used for p2p stuff)            &                                                              \\ \hline
		 productid  & An ID that identify the game you're using            &                                                                                                                                                     \\ \hline
		 gamename   & A string that rapresents the game that you're using, used also for several activities like peerchat server identification             &                                                                    \\ \hline
		
		namespaceid & ?   &                                                                                                                                                                                                      \\ \hline
		sdkrevision & The version of the SDK you're using& \\ \hline
		   quiet    & ? Maybe indicate invisible login which can not been seen at friends list &\\ \hline
		   id& The value is 1&\\ \hline
		   final & Message end&\\ \hline
	\end{tabular} 
	\caption{Login parameter string}
	\label{Login parameter string}
\end{table}
\paragraph{Login Response From Server}\mbox{}\\

This response string \ref{Login response string1}, \ref{Login response string2} is send by the server when a connection is accepted, and followed by a challenge\ref{Chanllenge string}, which verifies the server that client connect to.
\par There are two kinds of login response string: 
\begin{equation}\label{Login response string1}
\begin{split}
&\backslash lc \backslash 1 \backslash challenge \backslash \langle challenge \rangle \backslash nur \\&\backslash userid \backslash \langle userid \rangle \backslash profileid \backslash \langle profileid \rangle \backslash final \backslash
\end{split}	
\end{equation}

\begin{table}[H]
	\centering
	\begin{tabular}{A}

		\hline 
		\textbf{Keys}&\textbf{Description}&\textbf{Type}  \\ 
		\hline 
		challenge & The challenge string sended by GameSpy Presence server&  \\ 		
		\hline 
		nur & ? &\\
		\hline 
 userid&The userID of the profile &\\	\hline 
 profileid&The profileID &\\	\hline 
 final& &\\	\hline 
	\end{tabular} 
	\caption{The first type login response}
	\label{The first type login response}	
\end{table}

\begin{equation}\label{Login response string2}
\begin{split}
&\backslash lc \backslash 2 \backslash sesskey \backslash \langle sesskey \rangle  \backslash userid \backslash \langle userid \rangle \backslash profileid \backslash \langle profileid \rangle \\& \backslash uniquenick \backslash \langle uniquenick \rangle \backslash lt \backslash \langle lt \rangle \backslash proof \backslash \langle proof \rangle \backslash final \backslash
\end{split}	
\end{equation}

\begin{table}[H]
	\centering
	\begin{tabular}{A}
		\hline 
		\textbf{Keys}& \textbf{Description}&\textbf{Type}  \\ 
		\hline 
		sesskey & The session key, which is a integer rapresentating the client connection& \\ 		
		\hline 
		userid & The userID of the profile& \\
		\hline 
		profileid&The profileID &\\	\hline 
		uniquenick&The logged in unique nick &\\	\hline 
		lt& The login ticket, unknown usage&\\\hline
		proof& The proof is something similar to the response but it vary&\\\hline
		final& &\\
	\hline 
	\end{tabular} 
	\caption{The second type login response}
	\label{The second type login response}
\end{table}
Proof in \ref*{The second type login response} generation: $ md5(password)||48 spaces $
The user could be AuthToken or the User/UniqueNick (with the extra PartnerID).
server challenge that we received before.
the client challenge that was generated before.




\subsubsection{User Creation}
This commmand \ref{Create user command} is used to create a user in GameSpy.
\begin{equation}\label{Create user command}
\begin{split}
&\backslash newuser \backslash email \backslash \langle email \rangle \backslash nick \backslash \langle nick \rangle \\& \backslash passwordenc \backslash \langle passwordenc \rangle 
\backslash productid \backslash \langle productid \rangle \\& \backslash gamename \backslash \langle gamename \rangle \backslash uniquenick \backslash \langle uniquenick \rangle \\& \backslash cdkeyenc \backslash \langle cdkeyenc \rangle \backslash partnerid \backslash \langle partnerid \rangle \backslash id \backslash 1 \backslash final \backslash
\end{split}	
\end{equation}
The description of each parameter string is shown in Table \ref{User creation string}.
\begin{table}[H]
	\centering
	\begin{tabular}{A}
		\hline
		  \textbf{String}    & \textbf{Description} &\textbf{Type}                                                                 \\ \hline
		   email    & The email used to create                                                    & \\ \hline
		   nick     & The nickname that will be created                                            & \\ \hline
		passwordenc & The encoded password (password XOR with Gamespy seed and the Base64 encoded)  &\\ \hline
		 productid  & An ID that identify the game you're using                                     &\\ \hline
		 gamename   & A string that rapresents the game that you're using, used also for several    &\\ \hline
		namespaceid & ?Unknown                                                                      &\\ \hline
		uniquenick  & Uniquenick that will be created                                               &\\ \hline
		 cdkeyenc   & The encrypted CDkey, encrypted method is the same as the passwordenc          &\\ \hline
		 partnerid  & This ID is used to identify a backend service logged with gamespy             &\\ \hline
		    id      & The value of id is 1                                                          &\\ \hline
		   final    & Message end                                                                   &\\ \hline
	\end{tabular}
\caption{User creation string}
\label{User creation string}
\end{table}

\section{GameSpy Presence Search Player}
Table \ref{IP and Port for GPSP} are the GPSP IP and Ports that client/game connect to.
\begin{table}[H]
	\centering
	\begin{tabular}{|c|c|}
		\hline 
		\textbf{IP}&\textbf{Port}  \\ 		
		\hline 
		gpsp.gamespy.com&29901 \\
		\hline 
	\end{tabular} 
	\caption{IP and Ports for GameSpy Presence Search Player}
	\label{IP and Ports for GameSpy Presence Search Player}
	\end{table}
\subsection{Search User}
This is the request that client sends to server:
\begin{equation}\label{Search User Request}
\begin{split}
 \backslash search\backslash\backslash sesskey \backslash<sesskey>\backslash profileid \backslash <profileid> \\ \backslash namespaceid\backslash <namespaceid>  \backslash partnerid\backslash <partnerid>\\ \backslash nick \backslash <nick> \backslash uniquenick \backslash <uniquenick> \\ \backslash email \backslash <email> \backslash gamename \backslash <gamename> \backslash final \backslash
\end{split}
\end{equation}

This is the response that server sends to client:
\begin{equation}
	\begin{split}
	\backslash bsr \backslash <profileid> \backslash nick \backslash <nick>	\backslash uniquenick \backslash <uniquenick> \\
	\backslash namespaceid \backslash <namespaceid>\backslash firstname \backslash <firstname> \\ 
	\backslash lastname \backslash <lastname>\backslash email \backslash <email> \\
	\backslash bsrdone \backslash <gamespy enc determinator> \backslash final \backslash
	\end{split}
\end{equation}


\part{RetroSpy System Architecture}



\chapter{Introduction}
2014 is a year that thousands games are abandon by GameSpy. I still remember the day when GameSpy shutdown, no matter how hard i try to login into Crysis2 multiplayer, but the same error shows "An error has occurred. Please check your network connectivity"\ref{Crysis2 error message}.
\begin{figure}[H]
	\centering
	\caption{Crysis2 error message}
	\label{Crysis2 error message}
	\includegraphics[scale=0.25]{Pictures/RetroSpyLogoWithText.png}
\end{figure}

\chapter{Database}
This kind of string is represent a parameter in a request sends by the client \ref{Parameter string}. GameSpy uses the combination of the parameter to search the string with value, and sends the data back to client use this kind of parameter string.
\subsection{Database Key Field}
These keys is that GameSpy Presence SDK using to find a user in their database. Keys are shown in Table \ref{Key Field}.

\begin{table}[H]
	\centering
	\begin{tabular}{|c|>{\centering\arraybackslash}p{8cm}|}
		\hline 
		\textbf{Keys}& \textbf{Description}  \\ 
		\hline 
		$user$ & An user contains the Email and the password, but contains multiple profiles \\ 		
		\hline 
		$profileid$ & The profile contains the name, surname, birth date and all the rest user info, including
		an unique nickname used to identify the profile and a generic nickname used to show for example in
		games \\
		\hline 
	\end{tabular} 
	\caption{Key Field}
	\label{Key Field}	
\end{table}

\begin{table}[H]
	\centering
	\begin{tabular}{|c|c|}
		\hline 
		\textbf{String}&\textbf{Description}  \\ 
		\hline 
		$ \backslash id \backslash 1 \backslash $& This is a parameter string the value of $ id $ is $ 1 $ \\ 		
		\hline 
		$ \backslash profileid \backslash 007 \backslash $ & This is a parameter string the value of $ profileid $ is $ 007 $ \\
		\hline
	\end{tabular} 
	\caption{Parameter string}
	\label{Parameter string}
\end{table}

\chapter{The Detail of RetroSpy Project}
All projects in the RetroSpy visual studio solution is listed as follows.
\begin{itemize}
	\item GameSpyLib: The library for all RetroSpy servers.
	\item CDKey: CD-Key server.
	\item NATNegotiation: NAT negotiation server.
	\item PresenceConnectionManager: GPCM server.
	\item PresenceSearchPlayer:  GPSP server.
	\item QueryReport: Query report server.
	\item ServerBrowser: Server browser server.
	\item StatsAndTracking: Stats and tracking server.
	\item SAKEPersistentStorage: SAKE persistent storage server.
\end{itemize}
\section{GameSpyLib}
	\subsection{Common}
	\subsection{Database}
	\subsection{Extensions}
	\subsection{Logging}
	\subsection{Networks}
	There are two different servers in RetroSpy; one is TCP another is UDP.  TCP and UDP work differently so the implementation will be different. We show the different implementing in \ref{Tcp} and \ref{Udp}.
	\subsubsection{Tcp}\label{Tcp}
	TcpServer class is only for making the connection and listening for connections. TcpStream is for receiving and sending the message.
	\subsubsection{Udp}\label{Udp}
	UdpServer class does not need a server to handle connection and listen for connection, every client can be a server, and every server is a client. So this class has both receiving and sending functions.

\section{CDKey}
\section{NATNegotiation}
\section{PresenceConnectionManager}
\subsection{LoginHandler}
Different game will have different request. Some of request contain different combination of <key,value> pairs. The request we already known is shown in the Table~\ref{Game requests key}.
\begin{table}[H]
	\centering
	\begin{tabular}{|c|c|}
		\hline 
		\textbf{Game name}&\textbf{request}  \\ 
		\hline 		
		armada2 & firewall,port \\
		\hline
		gslive& uniquenick,productid,partnerid,gamename,sdkrevision \\ 		
		\hline 
		Crysis2&uniquenick,productid,partnerid,gamename,sdkrevision \\
		\hline
	\end{tabular} 
	\caption{Game requests key}
	\label{Game requests key}
\end{table}
\section{PresenceSearchPlayer}
\subsection{NewUserHandler}

\section{QueryReport}
\section{ServerBrowser}
\section{StatsAndTracking}
\section{SAKEPersistentStorage}

\chapter{conclusion}

\end{comment}

\end{document}
