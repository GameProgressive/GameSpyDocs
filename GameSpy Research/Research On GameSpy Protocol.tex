\documentclass[oneside,titlepage,a4paper]{Definition/retrospy} %book,article,report,letter


\begin{document}

\title{\Huge\textbf{Research On GameSpy Protocol}} 
\author{Arves100, xiaojiuwo}


%\date{} %%如果没有这句,会生成时间

\maketitle  %%生成书名

\tableofcontents  %%生成目录

%\mainmatter %%表示文章的正文部分,在生成目录后将从第一页开始
\chapter{Introduction}

\section{The History of GameSpy}

\section{Related Works}


\chapter{General Construction}
In this chapter we describe the structure of GameSpy SDK and GameSpy servers.
\section{SDK Module}
GameSpy SDK contains of  16 modules.
	\begin{itemize}
		\item Brigades
		\item Chat
		\item Presence \& Messaging
		\item CDKey
		\item Stats \& Tracking
		\item Persistent Storage
		\item Transport
		\item NAT Negotation
		\item Peer to Peer communication
		\item Patching \& Tracking
		\item Server Browser
		\item Query \& Reporting
		\item SAKE Persistent Storage
		\item ATLAS Competition
		\item Voice Chat
		\item Web Authentication
	\end{itemize}

\section{GameSpy Backend Servers}
\par GameSpy backend servers are list as follows.
	\begin{itemize}
		\item GameSpy Presence Connection Manager (GPCM)
		\item GameSpy Presence Search Player(GPSP)
		\item GameSpy Query and Report (QR)
		\item GameSpy Chat
		\item GameSpy SAKE Storage (SAKE)
		\item GameSpy NAT Negotation (NatNeg)
		\item GameSpy Server Browser (SB)
		\item GameSpy CDKey 
		\item GameSpy Stats \& Tracking (GStats)
		\item GameSpy Web Services
	\end{itemize}

\section{The Access Sequence of Client}
If a user want to use GameSpy service, the access sequence is listed in Figure~\ref{The access sequence of client} and we describe the detail below.
\begin{figure}[H]
	\centering
	\caption{The access sequence of client}
	\label{The access sequence of client}
	\scalebox{0.7}{
	\begin{tikzpicture}  [node distance = 2cm, auto,transform shape]
	\node[block] (gpcm) {GPCM};
	%%%%%%%%%%%right nodes%%%%%%%%%%
	\node[block, right= of gpcm](gpsp){GPSP};
	\node[block, right= of gpsp](qr){QR};
	\node[block, below= of qr](sb){SB};
	%%%%%%%%%%%left nodes%%%%%%%%%%%
	\node[block,below= of gpcm](cdkey){CDKey};

		%%%%%%%%%%%bottom nodes%%%%%%%%
	\node[block, below= of gpsp] (client) {Client};  
	\node[block, below= of client](gstats){GSTATS};
	\node[block, right= of gstats](webserver){WebServer};
		\node[block, left= of gstats](chat){CHAT};
	%%%%%%%%%%%top nodes%%%%%%%%%%
	\node[block, above= of gpcm](gameserver){Game Server};
	
	
	%%%%%%%%%%%lines%%%%%%%%%%%%%
	\draw[line] (client)--node [midway, above, sloped] {\circled{\small{2}}}(gpcm);
	\draw[line](client)--node [midway, above, sloped] {\circled{\small{2}}}(gpsp);
	\draw[line](client)--node [midway, above, sloped] {\circled{\small{3}}}(cdkey);
	\draw[line](client)--node [midway, above, sloped] {\circled{\small{4}}}(chat);
	\draw[line](client)--node [midway, above, sloped] {\circled{\small{1}}} (qr);
	\draw[line](client)--node [midway, above, sloped] {\circled{\small{7}}}(sb);
	
	\draw[line](client)--node [midway, above, sloped] {\circled{\small{5}}}(gstats);
	\draw[line](client)--node [midway, above, sloped] {\circled{\small{5}}}(webserver);
	
	
	\draw[line](sb)--(qr);
	
	
	
	\draw[line](gameserver)--node [midway, above, sloped] {\circled{\small{6}}}(qr);
	\draw[line](client)--node [midway, above, sloped] {\circled{\small{8}}}(gameserver);
	\draw[line](gameserver.west) to [in=190,out=190] node [midway, above, sloped] {\circled{\small{9}}}(cdkey.west);
	
	
	
	
	%\node[block,below= of b] (c){GameSpy CDKey};
	% these are the name of the blocks
	
%	\draw[line] ([yshift=15pt]gpcm.west)--node [fill = white] {1.Send server challenge} ([yshift=15pt]client.east);
%	\draw[line] (client.east)--node [fill = white] {2.Send client challenge} (gpcm.west);
%	\draw[line] ([yshift=-15pt]gpcm.west)--node [fill = white] {3.Accept or reject} ([yshift=-15pt]client.east);
	
	\end{tikzpicture}  
}
\end{figure}
\paragraph{Access sequence explain}
\begin{enumerate}
	\item Client access to available check in QR server, which tells client GameSpy back-end server status.
	\item Client access GPCM or GPSP to check their account and login.
	\item Client access to CDKey to verify his cd-key in login phase.
	\item Client login to Chat server.
	\item Client retrieve player data(level, exp, etc.) from GStats(old game use this server to store player data, new game use Web Server to store player data).
	\item When a game server is launched it will send heartbeat to QR server to tell QR its information.
	\item Client access to SB to search online game server.
	\item Client login to game server with his information and cd-key.
	\item Game server will check his cd-key by accessing to CDKey server, after every information is verified, client should be able to play their game.
\end{enumerate}


\section{Basic Description of Protocol}
In this part, we describe some of the basic patterns that are used in all GameSpy servers.

\subsection{The String Pattern}
We first introduce the pattern of the string, which is using to make up a request and response.
The following servers do use the pattern: Presence Connection Manager, Presence Search Player, GameSpy Status and Tracking, CD-Key, Query Report(version 1)
This kind of string represents a value in a request and response sent by the client or the server as Table \ref{String pattern}.\\


\begin{table}[H]
	\centering
	\begin{tabular}{|c|c|}
		\hline 
		\textbf{String}&\textbf{Description}  \\ 
		\hline 
		$ \backslash key \backslash \langle value \rangle \backslash $& The key is $ key $, the value of the key is $  value  $  \\ 
 		\hline
	\end{tabular} 
	\caption{String pattern}
	\label{String pattern}
\end{table}
There are two kind of patterns the first one is value string, the second one is command string.
\paragraph{Value String}
This kind of string represents a key value pair in the request or response string, it has a key and a correspond value as shown in Table~\ref{Value string}.
\begin{table}[H]
	\centering
	\begin{tabular}{|c|c|}
		\hline 
		\textbf{String}&\textbf{Description}  \\ 
		\hline 
		$ \backslash pid \backslash  13  \backslash $& The key is $ pid $, the value of the $ pid $ is $  13  $  \\ 
		\hline
		$ \backslash userid \backslash  0  \backslash $& The key is $ userid $, the value of the $ userid $ is $  0  $  \\ 
		\hline
	\end{tabular} 
	\caption{Value string}
	\label{Value string}
\end{table}


\paragraph{Command String}

This kind of string represents a command in a request sends by the client or the server as Table \ref{Command string}.
The command will end with $ \backslash \backslash $ or $ \backslash $ depends on whether run at the server-side or client-side.


\begin{table}[H]
	\centering
	\begin{tabular}{|c|c|}
		\hline 
		\textbf{String}&\textbf{Description}  \\ 
		\hline 
		$ \backslash command \backslash\backslash $& This is a command \\ 		
		\hline
	\end{tabular} 
	\caption{Command string}
	\label{Command string}
\end{table}


\chapter{GameSpy Presence and Messaging}
\par Presence \& Messaging system allows a game to add account authentication or registration, which includes a profile where personal information could be stored (such as email, first name), a friend list (called buddies), private messages.
\par GameSpy Presence contains two server, GameSpy Presence Connection Manager (GPCM) and GameSpy Presence Search Player (GPSP).
GPCM is a server that manages the profiles (such as login, storing the profile information).

\section{Common Information}
In this section we describe the common information, methods, techniques that GPCM and GPSP have.
\subsection{Server IP and Ports}
Table \ref{IP and Ports for GameSpy Presence Servers} are the  IP and Ports of GPCM and GPSP that client or game connect to.
\begin{table}[H]
	\centering
	\begin{tabular}{|c|c|c|}
		\hline 
		\textbf{Name}&\textbf{IP}&\textbf{Port}\\ 
		\hline 
		GPCM&gpcm.gamespy.com&29900 \\ 
	 	\hline 
		GPSP&gpsp.gamespy.com&29901 \\
		\hline
	\end{tabular} 
\caption{IP and Ports for GameSpy Presence Servers}
\label{IP and Ports for GameSpy Presence Servers}

\end{table}

\section{GameSpy Presence Connection Manager}

\subsection{Communication Diagram}
\begin{figure}[H]
	\scalebox{0.8}{
	\begin{tikzpicture}
		\node[block] (a) {Client};  
		\node[block,right=200pt of a] (b) {GPCM};
		\draw[line] ([yshift=15]b.west)--node [fill = white] {1.Send server challenge} ([yshift=15]a.east);
		\draw[line] (a.east)--node [fill = white] {2.Send client challenge} (b.west);
		\draw[line] ([yshift=-15]b.west)--node [fill = white] {3.Accept or reject} ([yshift=-15]a.east);
	\end{tikzpicture}
}
\end{figure}


\subsection{Request For GameSpy Presence Connection Manager}
Table \ref{Request For GameSpy Presence Connection Manager} lists the request (known by us) that clients send to GameSpy Presence Connection Manager server (GPCM).
\begin{table}[H]
	\centering
	\begin{tabular}{B}
		\hline 
		\textbf{Commands}&\textbf{Description}  \\ 
		\hline 
		$\backslash inviteto \backslash\backslash$& Invite friends\\ 		
		\hline 
		$\backslash login \backslash\backslash$&Login to GPCM \\
		\hline
 		$\backslash getprofile \backslash\backslash$&	Get the profile of a player (including your own)\\
 		\hline
		$\backslash addbuddy \backslash\backslash$& Add a player to my friend list \\
		\hline
		$\backslash delbuddy \backslash\backslash$ & Delete a player from my friend list \\
		\hline
		$\backslash updateui \backslash\backslash$& Update login information (email, password) \\
		\hline
		$\backslash updatepro \backslash\backslash$& Update my profile such as first name, last name, gender etc. \\
		\hline
		$\backslash logout \backslash\backslash$& Logout manually by user\\
		\hline
		$\backslash status \backslash\backslash$& Update the status of a user (Such as what game is the player playing) \\
		\hline
		$\backslash ka \backslash\backslash$& Keep client alive (do not disconnect) \\

		\hline 
	\end{tabular} 
	\caption{Request For GameSpy Presence Connection Manager}
	\label{Request For GameSpy Presence Connection Manager}
\end{table}

Error response string for (GPCM, GPSP):
\begin{equation}
\begin{split}
\backslash error \backslash\backslash err \backslash \langle error code \rangle \backslash fatal\backslash\backslash errmsg \backslash \langle error message \rangle \backslash id\backslash 1 \backslash final \backslash
\end{split}	
\end{equation}
\subsubsection{Login Phase}
\myparagraph{Client Login Request}
There are three ways of login:
\begin{itemize}
	\item AuthToken: Logging using an alphanumeric string that rapresents an user
	\item 	UniqueNick: Logging using a nickname that is unique from all the players
	\item User: Logging with the nickname and the password
\end{itemize}

The full login request string:
\begin{equation}\label{Chanllenge string}
\begin{split}
	&\backslash login \backslash challenge \backslash \langle challenge \rangle \backslash authtoken \backslash \langle authtoken \rangle \\& \backslash uniquenick \backslash \langle uniquenick \rangle \backslash user \backslash \langle user \rangle 
	\backslash userid \backslash \langle userid \rangle \\& \backslash profileid \backslash \langle profileid \rangle \backslash partnerid \backslash \langle partnerid \rangle \backslash response \backslash \langle response \rangle \\&
	 \backslash firewall \backslash 1 \backslash port \backslash \langle port \rangle \backslash productid \backslash  \langle productid \rangle \\& \backslash gamename \backslash \langle gamename \rangle \backslash namespaceid \backslash \langle namespaceid \rangle \\& \backslash  sdkrevision \backslash \langle sdkrevision \rangle \backslash quiet \backslash \langle quiet \rangle \backslash id \backslash 1 \backslash final \backslash
\end{split}
\end{equation}
The value $ \langle challenge \rangle $ for $ \backslash challenge \backslash $ in \ref{Chanllenge string} is a 10 byte alphanumeric string.

The following Table \ref{Login parameter string} is a description of string used in login request, GameSpy can use these string to find value in database.
\begin{table}[H]
	\centering
\begin{tabular}{A}
		\hline
		\textbf{Keys} & \textbf{Description} & \textbf{Type}
			                                                                          \\ \hline
		 login& The login command which use to identify the login request of client&\\ \hline
		 challenge  & The user challenge used to verify the authenticity of the client     &                                                                                                                                     \\ \hline
		 authtoken  & The token used to login (represent of an user)        &                                                                                                                                                    \\ \hline
		uniquenick  & The unique nickname used to login       &                                                                                                                                                                  \\ \hline
		   user     & The users account (format is NICKNAME@EMAIL)           &                                                                                                                                                   \\ \hline
		  userid    & Send the userid (for example when you disconnect you will keep this)              &                                                                                                                        \\ \hline
		 profileid  & Send the profileid (for example when you disconnect you will keep this)           &                                                                                                                        \\ \hline
		 partnerid  & This ID is used to identify a backend service logged with gamespy.(Nintendo WIFI Connection will identify his partner as 11, which means that for gamespy, you are logging from a third party connection) &\\ \hline
		 response   & The client challenge used to verify the authenticity of the client     &                                                                                                                                   \\ \hline
		 firewall   & If this option is set to 1, then you are connecting under a firewall/limited connection &\\
		 \hline
		 port& The peer port (used for p2p stuff)            &                                                              \\ \hline
		 productid  & An ID that identify the game you're using            &                                                                                                                                                     \\ \hline
		 gamename   & A string that rapresents the game that you're using, used also for several activities like peerchat server identification             &                                                                    \\ \hline
		
		namespaceid & ?   &                                                                                                                                                                                                      \\ \hline
		sdkrevision & The version of the SDK you're using& \\ \hline
		   quiet    & ? Maybe indicate invisible login which can not been seen at friends list &\\ \hline
		   id& The value is 1&\\ \hline
		   final & Message end&\\ \hline
	\end{tabular} 
	\caption{Login parameter string}
	\label{Login parameter string}
\end{table}
\paragraph{Login Response From Server}\mbox{}\\

This response string \ref{Login response string1}, \ref{Login response string2} is send by the server when a connection is accepted, and followed by a challenge\ref{Chanllenge string}, which verifies the server that client connect to.
\par There are two kinds of login response string: 
\begin{equation}\label{Login response string1}
\begin{split}
&\backslash lc \backslash 1 \backslash challenge \backslash \langle challenge \rangle \backslash nur \\&\backslash userid \backslash \langle userid \rangle \backslash profileid \backslash \langle profileid \rangle \backslash final \backslash
\end{split}	
\end{equation}

\begin{table}[H]
	\centering
	\begin{tabular}{A}

		\hline 
		\textbf{Keys}&\textbf{Description}&\textbf{Type}  \\ 
		\hline 
		challenge & The challenge string sended by GameSpy Presence server&  \\ 		
		\hline 
		nur & ? &\\
		\hline 
 userid&The userID of the profile &\\	\hline 
 profileid&The profileID &\\	\hline 
 final& &\\	\hline 
	\end{tabular} 
	\caption{The first type login response}
	\label{The first type login response}	
\end{table}

\begin{equation}\label{Login response string2}
\begin{split}
&\backslash lc \backslash 2 \backslash sesskey \backslash \langle sesskey \rangle  \backslash userid \backslash \langle userid \rangle \backslash profileid \backslash \langle profileid \rangle \\& \backslash uniquenick \backslash \langle uniquenick \rangle \backslash lt \backslash \langle lt \rangle \backslash proof \backslash \langle proof \rangle \backslash final \backslash
\end{split}	
\end{equation}

\begin{table}[H]
	\centering
	\begin{tabular}{A}
		\hline 
		\textbf{Keys}& \textbf{Description}&\textbf{Type}  \\ 
		\hline 
		sesskey & The session key, which is a integer rapresentating the client connection& \\ 		
		\hline 
		userid & The userID of the profile& \\
		\hline 
		profileid&The profileID &\\	\hline 
		uniquenick&The logged in unique nick &\\	\hline 
		lt& The login ticket, unknown usage&\\\hline
		proof& The proof is something similar to the response but it vary&\\\hline
		final& &\\
	\hline 
	\end{tabular} 
	\caption{The second type login response}
	\label{The second type login response}
\end{table}
Proof in \ref*{The second type login response} generation: $ md5(password)||48 spaces $
The user could be AuthToken or the User/UniqueNick (with the extra PartnerID).
server challenge that we received before.
the client challenge that was generated before.




\subsubsection{User Creation}
This commmand \ref{Create user command} is used to create a user in GameSpy.
\begin{equation}\label{Create user command}
\begin{split}
&\backslash newuser \backslash email \backslash \langle email \rangle \backslash nick \backslash \langle nick \rangle \\& \backslash passwordenc \backslash \langle passwordenc \rangle 
\backslash productid \backslash \langle productid \rangle \\& \backslash gamename \backslash \langle gamename \rangle \backslash uniquenick \backslash \langle uniquenick \rangle \\& \backslash cdkeyenc \backslash \langle cdkeyenc \rangle \backslash partnerid \backslash \langle partnerid \rangle \backslash id \backslash 1 \backslash final \backslash
\end{split}	
\end{equation}
The description of each parameter string is shown in Table \ref{User creation string}.
\begin{table}[H]
	\centering
	\begin{tabular}{A}
		\hline
		  \textbf{String}    & \textbf{Description} &\textbf{Type}                                                                 \\ \hline
		   email    & The email used to create                                                    & \\ \hline
		   nick     & The nickname that will be created                                            & \\ \hline
		passwordenc & The encoded password (password XOR with Gamespy seed and the Base64 encoded)  &\\ \hline
		 productid  & An ID that identify the game you're using                                     &\\ \hline
		 gamename   & A string that rapresents the game that you're using, used also for several    &\\ \hline
		namespaceid & ?Unknown                                                                      &\\ \hline
		uniquenick  & Uniquenick that will be created                                               &\\ \hline
		 cdkeyenc   & The encrypted CDkey, encrypted method is the same as the passwordenc          &\\ \hline
		 partnerid  & This ID is used to identify a backend service logged with gamespy             &\\ \hline
		    id      & The value of id is 1                                                          &\\ \hline
		   final    & Message end                                                                   &\\ \hline
	\end{tabular}
\caption{User creation string}
\label{User creation string}
\end{table}

\section{GameSpy Presence Search Player}
Table \ref{IP and Ports for GameSpy Presence Servers} are the GPSP IP and Ports that client/game connect to.

\subsection{Search User}
This is the request that client sends to server:
\begin{equation}\label{Search User Request}
\begin{split}
 \backslash search\backslash\backslash sesskey \backslash<sesskey>\backslash profileid \backslash <profileid> \\ \backslash namespaceid\backslash <namespaceid>  \backslash partnerid\backslash <partnerid>\\ \backslash nick \backslash <nick> \backslash uniquenick \backslash <uniquenick> \\ \backslash email \backslash <email> \backslash gamename \backslash <gamename> \backslash final \backslash
\end{split}
\end{equation}

This is the response that server sends to client:
\begin{equation}
	\begin{split}
	\backslash bsr \backslash <profileid> \backslash nick \backslash <nick>	\backslash uniquenick \backslash <uniquenick> \\
	\backslash namespaceid \backslash <namespaceid>\backslash firstname \backslash <firstname> \\ 
	\backslash lastname \backslash <lastname>\backslash email \backslash <email> \\
	\backslash bsrdone \backslash <gamespy enc determinator> \backslash final \backslash
	\end{split}
\end{equation}



\chapter{GameSpy Status and Tracking}
when game connect to GSTATS server, server will send an message to game which contains the challenge, the total length of message must bigger than 38bytes, and the challenge must bigger than 20bytes.
when game received the challenge it will compute a response, the response is formed as follows. 
response = CRC32(<server challenge>,<length of server challenge>)||<game secret key>
then game will compute the MD5 hash as MD5value = MD5(<response>,<length of response>)
then encoded with Enctype3
then construct the challenge-response message as $ \backslash auth \backslash \backslash gamename \backslash <gamename>\backslash response \backslash <MD5value> \backslash port \backslash <port> \backslash id \backslash <id> $

session key length (unknown)
connction id = transfer ascii of sessionkey to integer

the initialization phase is finished.
server challenge message length (bigger than 38-byte)
server challenge length (bigger than 20-byte)
$ \backslash final \backslash $ is encrypted using XOR Enctype1 at the end of the challenge that sends by the server.


\end{document}
