\documentclass[oneside,titlepage,a4paper]{report} %book,article,report,letter
\usepackage{amsmath} %宏包
\usepackage{graphicx}
\usepackage{amssymb}
\usepackage{titling}
\usepackage{hyperref}
\usepackage{pdfpages}
\usepackage{xcolor}
\usepackage{enumerate}
\usepackage{float}
\usepackage{multirow} 
\usepackage{shorttoc}
\usepackage{tabularx}
\usepackage{geometry}
%\geometry{left=2.0cm,right=2.0cm,top=2.0cm,bottom=2.0cm}
%\usepackage[T1]{fontenc}
%\usepackage{makecell}
%\renewcommand\theadfont{\bfseries\sffamily}
%导言区

\setcounter{tocdepth}{4}
\setcounter{secnumdepth}{3}
\begin{document}

\title{\Huge\textbf{Research On GameSpy Protocol}} 
\author{Arves100, Xiaojiuwo}

\renewcommand\maketitlehooka{
	\pagestyle{empty}
	\centering
	}
\renewcommand\maketitlehookc{	
	\includegraphics[width=5cm]{Pictures/GameSpylogoindustries.pdf}
	\vfill First Edition}


%\date{} %%如果没有这句,会生成时间

\maketitle  %%生成书名

\tableofcontents  %%生成目录

%\mainmatter %%表示文章的正文部分,在生成目录后将从第一页开始

\part{Research On GameSpySDK}

\chapter{GameSpy General Construction}
\par In GameSpy SDK there are 16 modules, which constructed the GameSpy main functions.
\section{GameSpy SDK Module}
\begin{itemize}
	\item GameSpy Presence Servers
	\begin{itemize}
		\item GameSpy Presence Connection Manager
		\item GameSpy Presence Search Player
	\end{itemize}
	\item Nat Negotiation
	\item Master Server: Query Report 2
	\item Master Server: Server Browser
	\item Master Server: Available Check
	\item Game Patching
	\item Game Tracking
	\item Master Server Patching: Downloading files from FilePlanet
	\item Peer SDK
	\item Game Statitics
	\item Chat Server
\end{itemize}

\chapter{GameSpy Presence Servers}
\par GameSpy Presence Servers contain two server, GameSpy Presence Connection Manager (GPCM) and GameSpy Presence Search Player (GPSP). GPCM is a server that handle login request and response with corresponding user infomation stored on GameSpy. GPSP is a server that handle search request for user.
\subsection{Server IP and Ports}
Table \ref{IP and Ports for GameSpy Presence Servers} are the GPCM and GPSP IP and Ports that client/game connect to.
\begin{table}[H]
	\centering
	\begin{tabular}{|c|c|}
		\hline 
		IP&Port  \\ 
		\hline 
		gpcm.gamespy.com&29900 \\ 		
		\hline 
	 	gpsp.gamespy.com&29901 \\
	 	\hline 
	\end{tabular} 
\caption{IP and Ports for GameSpy Presence Servers}
\label{IP and Ports for GameSpy Presence Servers}

\end{table}

\subsection{Database Key Field}
These keys is that GameSpy Presence SDK using to find a user in their database. Keys are shown in Table \ref{Key Field}.

\begin{table}[H]
	\centering
	\begin{tabular}{|c|>{\centering\arraybackslash}p{8cm}|}
		\hline 
		Keys& Description  \\ 
		\hline 
		User & An user contains the Email and the password, but contains multiple profiles \\ 		
		\hline 
		ProfileID & The profile contains the name, surname, birth date and all the rest user info, including
		an unique nickname used to identify the profile and a generic nickname used to show for example in
		games \\
		\hline 
	\end{tabular} 
	\caption{Key Field}
	\label{Key Field}	
\end{table}

\subsection{Protocol Descriptions}

In this part, we show the protocol detail in GameSpy Presence SDK.
\subsubsection{The Pattern of String}
We first introduce the pattern of the string, which is using to make up a request.
This kind of string is represent a value in a request sends by the client as Table \ref{Value string}.

\begin{table}[H]
	\centering
	\begin{tabular}{|c|c|}
		\hline 
		String&Description  \\ 
		\hline 
	$ \backslash \langle content \rangle \backslash $& The value is $ \langle content \rangle $  \\ 
		\hline 
	\end{tabular} 
\caption{Value string}
\label{Value string}
\end{table}

This kind of string is represent a command in a request sends by the client as Table \ref{Command string}.


\begin{table}[H]
	\centering
	\begin{tabular}{|c|c|}
		\hline 
		String&Description  \\ 
		\hline 
		$ \backslash command \backslash\backslash $& This is a command \\ 		
		\hline 
		$ \backslash error \backslash \backslash $ & Error command \\
		 \hline
		 $\backslash lc \backslash$& Login command\\
		 \hline
	\end{tabular} 
	\caption{Command string}
	\label{Command string}
\end{table}

This kind of string is represent a parameter in a request sends by the client \ref{Parameter string}. GameSpy uses the combination of the parameter to search the string with value, and sends the data back to client use this kind of parameter string.

\begin{table}[H]
	\centering
	\begin{tabular}{|c|c|}
		\hline 
		String&Description  \\ 
		\hline 
		$ \backslash id \backslash 1 \backslash $& This is a parameter string the value of $ id $ is $ 1 $ \\ 		
		\hline 
		$ \backslash profileid \backslash 007 \backslash $ & This is a parameter string the value of $ profileid $ is $ 007 $ \\
		\hline
	\end{tabular} 
	\caption{Parameter string}
	\label{Parameter string}
\end{table}

Error response string for (GPCM, GPSP):
\begin{equation}
\begin{split}
\backslash error \backslash\backslash err \backslash \langle error code \rangle \backslash fatal\backslash\backslash errmsg \backslash \langle error message \rangle \backslash id\backslash 1 \backslash final \backslash
\end{split}	
\end{equation}
\subsubsection{Login Phase}
There are three ways of login:
\begin{itemize}
	\item AuthToken: Logging using an alphanumeric string that rapresents an user
	\item 	UniqueNick: Logging using a nickname that is unique from all the players
	\item User: Logging with the nickname and the password
\end{itemize}


Login response string: 
\begin{equation}\label{Login response string}
	\begin{split}
		\backslash lc \backslash 1
	\end{split}	
\end{equation}
This response string \ref{Login response string} is send by the server when a connection is accepted, and followed by a challenge\ref{Chanllenge string}, which verifies the server that client connect to.

The challenge string:
\begin{equation}\label{Chanllenge string}
	\backslash challenge \backslash \langle challenge \rangle
\end{equation}
The $ \langle challenge \rangle $ in \ref{Chanllenge string} is a 10 byte alphanumeric string.
\chapter{summary}

\chapter{introduction3}

\part{RetroSpy System Architecture}

\chapter{introduction}

\chapter{conclusion}

\end{document}
